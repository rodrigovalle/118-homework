\documentclass{report}
\usepackage{homework}
\solstrue

\renewcommand{\hmwkTitle}{Homework 8}

% Discussion
%

\begin{document}
\mktitle

\clearpage
\begin{problem}

List link layer functions.  For each of the function, list categories (if
applicable) and protocol/algorithm examples.

\begin{answer}{42em}
\begin{enumerate}
  \item Data Framing: the beginning and end of transmission chunks must be marked
    \begin{itemize}
      \item HDLC Bit Stuffing
      \item PPP Byte Stuffing
      \item Consistent Overhead Byte Stuffing (COBS)
    \end{itemize}

  \item Error Detection: if there were transmission errors, they must be detected
        and optionally corrected.
    \begin{itemize}
      \item Hamming Code (detection and correction)
      \item Parity bit (Single bit for error detection, or two
            dimensional to correct single bit errors).
      \item Cyclic Redundancy Check (CRC)
      \item Two-out-of-five Code (error detection)
      \item Repetition, e.g., repeat each bit three times. (error detection and
            correction)
      \item Checksums (detection): usually not provided by the link layer but
            rather the network (e.g. IP) and transmission (e.g. UDP) layers
            above it.
    \end{itemize}

  \item Channel Access Protocols \\
    Mostly concerned with broadcast mediums where everyone must share.
    \begin{enumerate}
      \item Channel Partitioning
        \begin{itemize}
          \item Time Division Multiple Access (TDMA): separate the channel
                into slots for each host, you can decide to use your slot or not.
          \item Frequency Division Multiple Access (FDMA): put everyone on a
                different frequency or employ frequency hopping techniques.
        \end{itemize}

      \item Channel Allocation
        \begin{enumerate}
          \item Central Allocator: one machine decides who gets to talk
          \item Token Passing: you get to talk if you're holding the token
        \end{enumerate}

      \item Random Access: talk whenever you want (when or be smart and talk
        when no one else is talking)
        \begin{itemize}
          \item ALOHA, Slotted ALOHA
          \item CSMA/CD, CSMA/CA
        \end{itemize}
    \end{enumerate}
\end{enumerate}
\end{answer}
\end{problem}



\clearpage
\begin{problem}

Consider two data sequences that are received from the network layer by the link
layer processor.

\begin{itemize}
\item \texttt{0x00 0x01 0x02 0xFF 0x7E 0x7D 0x00 0x00 0x7D 0xFD 0x7E}
\item \texttt{0x02 0x00 0x01 0x02 0x7D 0x00}
\end{itemize}

Create frame-encoded output (i.e., each input sequence should be encoded as a
separate frame) using
\begin{enumerate}
\item HDLC byte stuffing
\item PPP byte stuffing
\item COBS
\end{enumerate}


\begin{answer}{45em}
  \begin{enumerate}
    \item HDLC
      \begin{itemize}
        \item \texttt{01111110 00000000 00000001 00000010 11111011 10111110 10011111 00100000 00000000 00001111 10011111 01101011 11101001 111110}
        \item \texttt{01111110 00000010 00000000 00000001 00000010 01111100 10000000 00111111 0}

      \end{itemize}

    \item PPP
      \begin{itemize}
        \item \texttt{0x7e 0x00 0x01 0x02 0xff 0x7d 0x5e 0x7d 0x5d 0x00 0x00 0x7d 0x5d 0xfd 0x7d 0x5e 0x7e}
        \item \texttt{0x7e 0x02 0x00 0x01 0x02 0x7d 0x5d 0x00 0x7e}
      \end{itemize}

    \item COBS: (Wikipedia says not to start with a \texttt{0x00} byte)
      \begin{itemize}
        \item \texttt{0x01 0x06 0x01 0x02 0xff 0x7e 0x7d 0x01 0x04 0x7d 0xfd 0x7e 0x00}
        \item \texttt{0x02 0x02 0x04 0x01 0x02 0x7d 0x01 0x00}
      \end{itemize}
  \end{enumerate}
\end{answer}
\end{problem}


\clearpage
\begin{problem}

Remember that Carrier Sensing Multiple Access (CSMA) protocols has several modes
of operation, including 1-persistent, non-persistent, p-persistent.

For each of the following protocols, which sensing mode is used.  For each,
describe the rationale for the choice.

\begin{enumerate}
\item ALOHA
\item Slotted ALOHA
\item bus-based Ethernet (e.g., old 10base2)
\item WiFi (e.g., 802.11abn)
\end{enumerate}

\begin{answer}{45em}
  \begin{enumerate}
    \item N/A:\@ ALOHA is not a CSMA protocol as it does not try to sense whether
          the medium is busy before attempting to transmit. Rather, ALOHA shoots
          first and asks questions later; when a collision is detected it will
          try to retransmit the frame again with some probability.

    \item N/A:\@ Slotted ALOHA is not CSMA in for the same reason as ALOHA, except
          on collision it tries to retransmit the frame with some probability in
          in each successive time slot.

    \item 1-persistent: Ethernet uses CSMA/CD;\@ when the medium is sensed busy,
          it will wait until it is idle and retransmit in the first avialable
          slot where this condition is met. This aggressive approach makes sense
          for bus Ethernet, where detecting a collision can be done quickly
          while transmitting. When such a collision is detected it can be
          cancelled immediately by transmitting a jam signal without having to
          complete the transmission. Because handling collisions on Ethernet is
          takes little time, it makes sense to try and transmit as soon as
          possible and possibly incur a small penalty rather than adopt and
          excessively conservative approach.

    \item p-persistent: WiFi uses CSMA/CA; when a the medium is sensed busy,
          it will wait until the medium is idle and try again with some
          probability. This conservative approach makes sense for WiFi where
          detecting and resolving packet collisions is significantly more
          expensive because packets are transmitted in their entirety. This is
          for two reasons: it's very difficult to listen for other transmissions
          while you're transmitting (your transmission will annihilate any other
          signals you might hear), and it's possible that your transmission
          could be colliding with another tranmission at the receiver that you
          can't hear.

  \end{enumerate}
\end{answer}
\end{problem}

\clearpage

% First you gotta broadcast
% A and router and switch 
% B's mac address
%
% who replies back: A, dst MAC addr is 
% switch and B can get the fram
% A is a new entry in the switch table
%
% B sends the IP packet to A, now B knows A's MAC address
% A source B, devices switch and A
% new entries: no
%
% ---
%
% how to send from B to node C

\begin{problem}

Consider the following network topology with specified MAC addresses for network interfaces and the configured IP addresses:

\begin{center}
  \includegraphics[scale=0.7]{hw8-2.pdf}
\end{center}

Assume the network mask for both subnetworks is 255.255.255.0.
\begin{enumerate}

  \item Assume that routing tables are properly configured and the network just
    started (i.e., all caches are empty), fill the following table to enumerate
    Ethernet frames (in chronological order) needed for node B to send an IP
    packet to 192.168.0.2 and receive a response back.

  \begin{table}[H]
    \centering
    \begin{tabular*}{1.0\textwidth}{c | c | c | c | c}
      \hline
      frame \# & dst MAC addr               & src MAC addr               & \pbox{15cm}{device (s) that can get the \\frame, excluding the sender} & \pbox{20cm}{new entries added into \\the switch's table (if any)} \\ \hline
      1        & \texttt{FF:FF:FF:FF:FF:FF} & \texttt{00:00:00:00:00:03} & Switch, Node A, Router                                                 & \texttt{00:00:00:00:00:03} $\rightarrow$ iface 3 \\  
      2        & \texttt{00:00:00:00:00:03} & \texttt{00:00:00:00:00:02} & Switch, Node B                                                         & \texttt{00:00:00:00:00:02} $\rightarrow$ iface 2 \\ 
      3        & \texttt{00:00:00:00:00:02} & \texttt{00:00:00:00:00:03} & Switch, Node A                                                         &                                               \\ 
      4        & \texttt{00:00:00:00:00:03} & \texttt{00:00:00:00:00:02} & Switch, Node B                                                         &                                               \\ 
               &                            &                            &                                                                        &                                               \\ 
               &                            &                            &                                                                        &                                               \\ 
               &                            &                            &                                                                        &                                               \\ 
               &                            &                            &                                                                        &                                               \\ 
               &                            &                            &                                                                        &                                               \\ 
               &                            &                            &                                                                        &                                               \\ 
               &                            &                            &                                                                        &                                               \\ 
               &                            &                            &                                                                        &                                               \\ 
               &                            &                            &                                                                        &                                               \\ 
               &                            &                            &                                                                        &                                               \\ 
               &                            &                            &                                                                        &                                               \\ 
               &                            &                            &                                                                        &                                               \\ 
               &                            &                            &                                                                        &                                               \\ 
               &                            &                            &                                                                        &                                               \\ 
               &                            &                            &                                                                        &                                               \\ 
    \end{tabular*}
  \end{table}

\clearpage
  \item Assume that the previous operation is done, fill the following table to
    enumerate Ethernet frames (in chronological order) for node B to send a
    packet to 192.168.1.253 and receive a reply.
  \begin{table}[H]
    \centering
    \begin{tabular*}{1.0\textwidth}{c | c | c | c | c}
      \hline
      frame \# & dst MAC addr               & src MAC addr               & \pbox{15cm}{device (s) that can get the\\ frame, excluding the sender} & \pbox{20cm}{new entries added into \\the switch's table (if any)} \\ \hline
      1        & \texttt{FF:FF:FF:FF:FF:FF} & \texttt{00:00:00:00:00:03} & Switch, Router, Node A                                                 &                                                  \\ 
      2        & \texttt{00:00:00:00:00:03} & \texttt{00:00:00:00:00:01} & Switch, Node B                                                         & \texttt{00:00:00:00:00:01} $\rightarrow$ iface 1 \\ 
      3        & \texttt{00:00:00:00:00:01} & \texttt{00:00:00:00:00:03} & Switch, Router                                                         &                                                  \\ 
      4        & \texttt{FF:FF:FF:FF:FF:FF} & \texttt{00:00:00:00:00:04} & Node C                                                                 &                                                  \\ 
      5        & \texttt{00:00:00:00:00:04} & \texttt{00:00:00:00:00:05} & Router                                                                 &                                                  \\ 
      6        & \texttt{00:00:00:00:00:05} & \texttt{00:00:00:00:00:04} & Node C                                                                 &                                                  \\ 
      7        & \texttt{00:00:00:00:00:04} & \texttt{00:00:00:00:00:05} & Router                                                                 &                                                  \\ 
      8        & \texttt{00:00:00:00:00:03} & \texttt{00:00:00:00:00:01} & Switch, Node B                                                         &                                                  \\ 
               &                            &                            &                                                                        &                                                  \\ 
               &                            &                            &                                                                        &                                                  \\ 
               &                            &                            &                                                                        &                                                  \\ 
               &                            &                            &                                                                        &                                                  \\ 
               &                            &                            &                                                                        &                                                  \\ 
               &                            &                            &                                                                        &                                                  \\ 
               &                            &                            &                                                                        &                                                  \\ 
               &                            &                            &                                                                        &                                                  \\ 
               &                            &                            &                                                                        &                                                  \\ 
               &                            &                            &                                                                        &                                                  \\ 
               &                            &                            &                                                                        &                                                  \\ 
               &                            &                            &                                                                        &                                                  \\ 
               &                            &                            &                                                                        &                                                  \\ 
               &                            &                            &                                                                        &                                                  \\ 
    \end{tabular*}
  \end{table}
\end{enumerate}

\end{problem}

\end{document}
