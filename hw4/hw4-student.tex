\documentclass{report}
\usepackage{homework}
\solstrue

\definecolor{mygray}{gray}{0.95}
\usepackage{listings}
\lstset{
basicstyle=\small\ttfamily,
columns=flexible,
breaklines=true,
backgroundcolor = \color{mygray},
framexleftmargin = 1em,
xleftmargin = 1em
}

\renewcommand{\hmwkTitle}{Homework 4}

\begin{document}
\mktitle


\begin{problem}

Below is an IP packet captured by Wireshark.
Bytes are shown in HEX mode, grouped by 4 HEX digits.
We know that this packet is using IPv4 with no option field.

\begin{lstlisting}
__ 00  00 1e
a7 f8  00 00
ff 11  __ __
ac 1f  09 ee
80 61  80 01
d4 2d  00 35
00 0a  __ __
61 61
\end{lstlisting}

\begin{enumerate}
\item What should be the HEX value for the first blank (line 1)?
\item What should be the HEX value for the second blank (two HEX digits, line
      3)?
\item What should be the HEX value for the third blank (two HEX digits, line
      7)?
\item What is the transport layer protocol carried in this IP packet? How do
      you know that?
\item What is the payload carried in the transport layer protocol you
      identified?
\end{enumerate}

\begin{answer}{30em}
  \begin{enumerate}
    \item The first should be 0x45, for IPv4 and a header length of
          $5\times4 = 20$ bytes.

    \item The second blank is the IP header checksum, which is calculated as
          follows:\\
            \texttt{0x4500 + 0x001E + // version, IHL, <unused>, total length} \\
            \texttt{0xA7F8 + 0x0000 + // identification, flags, fragment off} \\
            \texttt{0xFF11 + 0x0000 + // TTL, protocol, checksum (zero)} \\
            \texttt{0xAC1F + 0x09EE + // source IP address} \\
            \texttt{0x8061 + 0x8001 = 0x3A296 // dest IP address}

          With wraparound: \texttt{0xA296 + 0x0003 = 0xA299}\\
          With bit flip: \texttt{0x5D66}

    \item The third blank is the UDP checksum:\\
            \texttt{0xAC1F + 0x09EE + // source IPv4 address}\\
            \texttt{0x8061 + 0x8001 + // destination IPv4 address}\\
            \texttt{0x0011 + 0x000A + // zeroes, UDP protocol, UDP length } \\
            \texttt{0xD42D + 0x0035 + // source port, destination port } \\
            \texttt{0x000A + 0x0000 + // length, checksum (zero)} \\
            \texttt{0x6161 = 0x2EC57  // data}

          With wraparound: \texttt{0xEC57 + 0x0002 = 0xEC59}\\
          With bit flip: \texttt{0x13A6}

    \item The protocol is UDP, as the protocol field in the IP header reads
          $11_{16} = 17_{10}$, which is defined as UDP in RFC 790.

    \item The payload of this UDP packet is \texttt{0x6161}.
            
  \end{enumerate}
\end{answer}

\end{problem}

\clearpage
\begin{problem}

Assume that the pipeline sizes for the Go-back-N and TCP (no delayed ACKs)
protocols are large enough such that 5 consecutive data segments and their
corresponding ACKs can be received (if not lost in the channel) by the
receiving host (Host B) and the sending host (Host A) respectively.  Suppose
Host A sends 5 data segments to Host B, and the 2nd segment (sent from A) is
lost.  This loss is then detected by the protocol-specific means and the
protocol invokes recovery actions.  In the end, all 5 data segments are
correctly received by Host B.

\begin{enumerate}
\item In case of Go-back-N, how many segments has Host A sent in total and how
      many ACKs has Host B sent in total? 

\item In case of TCP (no delayed ACKs), how many segments has Host A sent in
      total and how many ACKs has Host B sent in total? 

\item If the timeout value happen to be set to 10 RTT, which of the protocol
      successfully delivers all five data segments in shortest time interval
      and give estimate of this time?
\end{enumerate}

\begin{answer}{30em}
  \begin{enumerate}
    \item In this case, Host A sends 5 segments to Host B but the second
          segment was lost. Host B sends 4 ACKs in response, all acknowledging
          the receipt of the first segment. After the timeout expires from not
          haveing received any ACKs for the segments in its current window,
          Host A resends segments \#2 through \#5 at which point Host B has
          received all 5 segments and ACKS segments \#2 through \#5. \\

          The number of segments sent is 5 + 4 = 9 and the number of ACKs sent
          is 4 + 4 = 8.

    \item In this case, Host A sends 5 segments to Host B, but segment \#2 is
          lost. Host B ACKs segment \#2 4 times, but buffers the other segments
          that it received out of order. Host A resends segment \#2 and Host B
          sends an ACK for segment \#6 indicating that it has already buffered
          segments \#3 - \#5.

          The number of segments sent is 5 + 1 = 6 and the number of ACKs sent
          is 4 + 1 = 5.

    \item If we make the simplifying assumption that we can send 5 packets on
          the network at the same time, and that TCP fast retransmit is set,
          TCP will take the least amount of time to deliver all 5 segments:

          1RTT (send first 5 segments simultaneously and ACK segment \#2 4
          times, triggering fast retransmit) + \\
          1RTT (send the missing packet \#2, it arrives in half a round trip and
          then Host B must acknowledge) =
          2RTT

          If we assume Go-back-N, we have to wait 1RTT for the first volley of
          packets, then we have to wait for the timeout to expire, and then
          we have to wait another RTT to resend the remaining 4 packets, for a
          grand total of 12 RTTs.
  \end{enumerate}
\end{answer}

\end{problem}

\clearpage
\begin{problem}

Host A and B are communicating over a TCP connection, and Host B has
already received from A all bytes up through byte 126. Suppose Host A then
sends two segments to Host B back-to-back. The first and second segments
contain 80 and 40 bytes of data, respectively. In the first segment, the
sequence number is 127, the source port number is 302, and the destination
port number is 80. Host B sends an acknowledgment whenever it receives a
segment from Host A.

\begin{enumerate}
\item In the second segment sent from Host A to B, what are the sequence
      number, source port number, and destination port number?

\item If the first segment arrives before the second segment, in the
      acknowledgment of the first arriving segment, what is the acknowledgment
      number, the source port number, and the destination port number?

\item If the second segment arrives before the first segment, in the
      acknowledgment of the first arriving segment, what is the acknowledgment
      number?

\end{enumerate}

\begin{answer}{24em}
  \begin{enumerate}
    \item sequence number: 207  \\
          source port: 302 \\
          destination port: 80
    \item acknowledgement number: 207 \\
          source port: 80 \\
          destination port: 302
    \item acknowledgement number: 127
  \end{enumerate}
\end{answer}

\end{problem}


\clearpage
\begin{problem}

What are the three charms in \emph{Harry Potter} that corresponds to our three
project names? What are their effects, respectively?

\begin{enumerate}
\item Accio (Project 1)
\item Confundo (Project 2)
\item Riddikulus (Project 3)
\end{enumerate}

\begin{answer}{24em}
  \begin{enumerate}
    \item ``This charm summons an object to the caster, potentially over
           a significant distance.''
    \item ``Causes the victim to become confused, befuddled, overly forgetful, and
          prone to follow simple orders without thinking about them.''
    \item ``A spell used when fighting a Boggart, 'Riddikulus' forces the
          Boggart to take the appearance of an object upon which the caster is
          concentrating.''
  \end{enumerate}
  Source: \url{https://en.wikipedia.org/wiki/List_of_spells_in_Harry_Potter}
\end{answer}

\end{problem}


\end{document}
