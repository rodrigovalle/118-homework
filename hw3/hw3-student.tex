\documentclass{report}
\usepackage{homework}
\solstrue

\definecolor{mygray}{gray}{0.95}
\usepackage{listings}
\lstset{
basicstyle=\small\ttfamily,
columns=flexible,
breaklines=true,
backgroundcolor = \color{mygray},
framexleftmargin = 1em,
xleftmargin = 1em
}

\renewcommand{\hmwkTitle}{Homework 3}

\begin{document}
\mktitle


\begin{problem}

Suppose you have a new computer just set up. \verb|dig| is one of the most
useful DNS lookup tool.
You can check out the manual of \verb|dig| at \url{http://linux.die.net/man/1/dig}.
A typical invocation of \verb|dig| looks like:
\verb|dig @server name type|.

Suppose that on April 19, 2017 at 15:35:21, you have issued
``\verb|dig google.com a|'' to get an IPv4 address for \url{google.com} domain
from your caching resolver and got the following result:

\begin{lstlisting}

; <<>> DiG 9.8.3-P1 <<>> google.com
;; global options: +cmd
;; Got answer:
;; ->>HEADER<<- opcode: QUERY, status: NOERROR, id: 17779
;; flags: qr rd ra; QUERY: 1, ANSWER: 1, AUTHORITY: 4, ADDITIONAL: 4

;; QUESTION SECTION:
;google.com.			IN	A

;; ANSWER SECTION:
google.com.		239	IN	A	172.217.4.142

;; AUTHORITY SECTION:
google.com.		55414	IN	NS	ns4.google.com.
google.com.		55414	IN	NS	ns2.google.com.
google.com.		55414	IN	NS	ns1.google.com.
google.com.		55414	IN	NS	ns3.google.com.

;; ADDITIONAL SECTION:
ns1.google.com.		145521	IN	A	216.239.32.10
ns2.google.com.		215983	IN	A	216.239.34.10
ns3.google.com.		215983	IN	A	216.239.36.10
ns4.google.com.		215983	IN	A	216.239.38.10

;; Query time: 81 msec
;; SERVER: 128.97.128.1#53(128.97.128.1)
;; WHEN: Wed Apr 19 15:35:21 2017
;; MSG SIZE  rcvd: 180

\end{lstlisting}

\begin{enumerate}

\item What is the discovered IPv4 address of \url{google.com} domain?

\item If you issue the same command 1 minute later, how would
      ``ANSWER SECTION'' look like?

\item When would be the earliest (absolute) time the caching resolver would
      contact one of the \url{google.com} name servers again?

\item When would be the earliest (absolute) time the caching resolver would
      contact one of the \url{.com} name servers?

\end{enumerate}


\begin{answer}{50em}
  \begin{enumerate}
  \item 172.217.4.142
  \item Exactly the same; the TTL is 239 seconds, so in 60 seconds the caching
        resolver will just serve the IP address from cache because the record
        hasn't expired yet.
  \item 239 seconds -- that's when the google.com A record expires from cache
        and the Google nameservers must be contacted again for the A record.
  \item 55414 seconds -- in this amount of time the authoritative nameserver
        records for \texttt{google.com} will expire from cache and the
        \texttt{.com} nameservers would have to be contacted again for these
        records. Note that the A records for these name servers in the
        \texttt{ADDITIONAL} section will still be in cache when this happens,
        but we still need valid nameserver records to verify that these
        A records belong to the authoritative nameservers of \texttt{google.com}.

  \end{enumerate}
\end{answer}

\end{problem}


\clearpage
\begin{problem}

In most of cases, we rely on caching resolvers to provide recursive DNS query
service for us. In this task, you will be a human caching resolver using
\verb|dig| command as your tool.

Look up an ``SRV'' resource record (a record that specifies the hostname and
port number of a server(s) for some service) for
\url{_ndn._udp.ucla.edu.ndn._homehub._autoconf.named-data.net}.

In your answer, include the exact commands you have used, including IP
addresses of the autoritative name servers to which you were sending DNS
queries, explain the returned result of each query (what is returned), and
indicate for how long you supposed to cache the returned information.

You can start with one of well-known IP addresses of the DNS root servers,
e.g., \url{198.41.0.4}.

\begin{answer}{30em}
  Assuming we have nothing in cache: \\
  \begin{enumerate}
  \item \texttt{dig @198.41.0.4 +norecurse <url> SRV} $\Rightarrow$ TTL 172800 seconds (valid in cache for this long) \\
        We query the root nameserver. This gives us a list of authoritative
        servers for the .net TLD.

  \item \texttt{dig @192.5.6.30 +norecurse <url> SRV} $\Rightarrow$ TTL 172800 seconds \\
        Now we ask the .net DNS servers if they have the SRV record for this
        URL. They give us the nameservers for named-data.net.

  \item \texttt{dig @129.82.138.8 +norecurse <url> SRV} $\Rightarrow$ TTL 86400 seconds \\
        Query the named-data.net name servers for the SRV record of our URL.
        They respond with the SRV record which contains: the hostname
        \texttt{spurs.cs.ucla.edu} and the port number 6363. They also respond
        with a list of authoritative nameserver hostnames which control the
        \texttt{\_autoconf.named-data.net} subdomain.

  \item NOTE: there could have been an extra query here. It just so happens
        that this particular name server had the SRV record we were looking for.
        If we had chosen a different nameserver from step 2 we might have had to
        send another query to \texttt{\_autoconf.named-data.net} nameservers to
        get our SRV record.
        (I tried it and this is what happened)
  \end{enumerate}

  The \texttt{+norecurse} option makes sure that these queries are one-off; that is, that none
  of these servers try to finish the query process for us recursively.
\end{answer}

\end{problem}

\clearpage
\begin{problem}

Suppose that you walked into Boelter Hall and get connected to \url{CSD} WiFi
network, which automatically gave you IP address of the local caching resolver.
However, initially, it doesn't allow you to do anything unless you type your
username and password in a popup window (or if you try to go to any website in
your browser).

\begin{enumerate}

\item Explain a mechanism of how does the ``\url{CSD}'' network achieve this /
      which features of DNS/HTTP make it possible.

\end{enumerate}

After you successfully logged in, you can start using the Internet. Suppose
the caching resolver has just rebooted and its cache is completely empty; RTT
between your computer and the caching resolver is $10 ms$ and RTT between the
caching resolver and any authoritative name server is $100 ms$; all responses
have TTL 12 hours.

\begin{enumerate}\addtocounter{enumi}{1}
\item If you try to go to \url{ucla.edu}, what would be minimum amount of time
  you will need to wait before your web browser will be able to initiate
  connect to the UCLA's web server?

\item What would be the time, if a minute later you will decide to go to
      \url{ccle.ucla.edu}?

\item What would be the time, if another minute later you will decide to go to
      \url{piazza.com}?

\item What would be the time, if another minute later you will decide to go to
      \url{gradescope.com}?

\end{enumerate}

\begin{answer}{30em}
  \begin{enumerate}
  \item The CSD network ``tricks'' you by giving the IP address of a cache
        resolver which resolves all non-allowed hostnames for non-whitelisted IPs
        (read: clients that aren't logged in) to the (local) IP address of the
        server hosting the login page. The cache resolver is able to do this
        because the client will implicitly trust that the cache resolver's
        response is accurate, so the cache resolver can resolve hostnames to
        whatever IP it likes. Then all the HTTP server has to do is serve the
        login page to the client, and whitelist the client's IP if the login
        was successful. Once whitelisted, the cache resolver will resolve
        hostnames for that requests from that IP normally.

  \item 10ms (query caching resolver) + 100ms (query root) +
        100ms (query .edu) + \\ 100ms (query ucla.edu nameserver for A record)
  \item 10ms (query caching resolver) + 100ms (query ucla.edu nameserver for A record)
  \item 10ms (query caching resolver) + 100ms (query root) +
        100ms (query .com) + \\ 100ms (query piazza.com nameserver for A record) \\
        (I tried this one using \texttt{dig} and the ucla.edu nameserver had an
        A record for ccle.ucla.edu)
  \item 10ms (query caching resolver) + 100ms (query .com) + \\
        100ms (query gradescope.com nameserver for A record)
  \end{enumerate}
\end{answer}


\end{problem}


\clearpage
\begin{problem}

  \begin{enumerate}
  \item An online chatting application is going viral.
        To optimize user experience, its developers decided to use CDN service
        to deliver superb chat application performance for clients around the
        world. What mechanisms CDN services use to help developers to do so?
        In your answer include specific mechanisms and basic idea how these
        mechanisms work.

  \item What are the factors that go into designing a CDN server selection
        strategy? Name at least four.
  \end{enumerate}

\begin{answer}{24em}
  \begin{enumerate}
  \item The CDN allows for rapid delivery of messages for clients that are
        geographically close to each other by connecting them both to a server
        that is roughly inbetween or near each of their locations. The issue
        with a traditional centralized server approach is, for example, if the
        server is in the US and the clients are in China, each message sent
        will impose the RTT from China to the US (assuming message push). This
        is clearly not ideal for clients in China. Instead, a CDN router can
        estimate the location of each client and connect them both to a server
        in their vicinity (or roughly inbetween them) in order to minimize the RTT.
  \item
    \begin{enumerate}
    \item Geographic location -- assume the caching resolver is geographically
          close to the client and take your best guess as to where the request
          is coming from. E.g. if the request is coming from 8.8.8.8, we know
          that IP is registered to Google Inc., which is located in Mountain
          View, so send this caching resolver a record with the IP of the Bay
          Area node and tell it to cache the record.

    \item Server availability/average response time (e.g. if the server for
          your region is under load, it might be faster to go to a server
          that's a little further away but more available)

    \item RTT (ping) (correlates with location) -- the routing node can
          estimate latency from various CDN server nodes to the client and chose
          the server with the lowest latency. If we cache these latencies it
          would be a big performance win (pinging takes an RTT) (or use Anycast).

    \item ISP cost -- it could cost more to serve from a node peered with a Tier
          1 ISP, and an alternative could provide comprable performance and
          cost less money for the CDN. It is a business, after all.
    \end{enumerate}

  \end{enumerate}
\end{answer}

\end{problem}


\end{document}
