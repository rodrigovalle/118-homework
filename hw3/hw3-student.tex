\documentclass{report}
\usepackage{homework}
\solstrue

\definecolor{mygray}{gray}{0.95}
\usepackage{listings}
\lstset{
basicstyle=\small\ttfamily,
columns=flexible,
breaklines=true,
backgroundcolor = \color{mygray},
framexleftmargin = 1em,
xleftmargin = 1em
}

\renewcommand{\hmwkTitle}{Homework 3}

\begin{document}
\mktitle


\begin{problem}

Suppose you have a new computer just set up. \verb|dig| is one of the most useful DNS lookup tool.
You can check out the manual of \verb|dig| at \url{http://linux.die.net/man/1/dig}.
A typical invocation of \verb|dig| looks like:
\verb|dig @server name type|.

Suppose that on April 19, 2017 at 15:35:21, you have issued ``\verb|dig google.com a|'' to get an IPv4 address for \url{google.com} domain from your caching resolver and got the following result:

\begin{lstlisting}

; <<>> DiG 9.8.3-P1 <<>> google.com
;; global options: +cmd
;; Got answer:
;; ->>HEADER<<- opcode: QUERY, status: NOERROR, id: 17779
;; flags: qr rd ra; QUERY: 1, ANSWER: 1, AUTHORITY: 4, ADDITIONAL: 4

;; QUESTION SECTION:
;google.com.			IN	A

;; ANSWER SECTION:
google.com.		239	IN	A	172.217.4.142

;; AUTHORITY SECTION:
google.com.		55414	IN	NS	ns4.google.com.
google.com.		55414	IN	NS	ns2.google.com.
google.com.		55414	IN	NS	ns1.google.com.
google.com.		55414	IN	NS	ns3.google.com.

;; ADDITIONAL SECTION:
ns1.google.com.		145521	IN	A	216.239.32.10
ns2.google.com.		215983	IN	A	216.239.34.10
ns3.google.com.		215983	IN	A	216.239.36.10
ns4.google.com.		215983	IN	A	216.239.38.10

;; Query time: 81 msec
;; SERVER: 128.97.128.1#53(128.97.128.1)
;; WHEN: Wed Apr 19 15:35:21 2017
;; MSG SIZE  rcvd: 180

\end{lstlisting}

\begin{enumerate}

\item What is the discovered IPv4 address of \url{google.com} domain?

\item If you issue the same command 1 minute later, how would ``ANSWER SECTION'' look like?

\item When would be the earliest (absolute) time the caching resolver would contact one of the \url{google.com} name servers again?

\item When would be the earliest (absolute) time the caching resolver would contact one of the \url{.com} name servers?

\end{enumerate}


\begin{answer}{50em}
  \begin{enumerate}
  \item 172.217.4.142
  \item Exactly the same; the TTL is 239 seconds, so in 60 seconds the caching
        resolver will just serve the IP address from cache because the record hasn't
        expired yet.
  \item 239 seconds (XXX)
  \item 145521 seconds -- the caching would contact the .com nameserver for the
        authoritative google nameserver once the ns1.google.com record expires.
  \end{enumerate}
\end{answer}

\end{problem}


\clearpage
\begin{problem}

In most of cases, we rely on caching resolvers to provide recursive DNS query service for us.
In this task, you will be a human caching resolver using \verb|dig| command as your tool.

Look up an ``SRV'' resource record (a record that specifies the hostname and port number of a server(s) for some service) for \url{_ndn._udp.ucla.edu.ndn._homehub._autoconf.named-data.net}.

In your answer, include the exact commands you have used, including IP addresses of the autoritative name servers to which you were sending DNS queries, explain the returned result of each query (what is returned), and indicate for how long you supposed to cache the returned information.

You can start with one of well-known IP addresses of the DNS root servers, e.g., \url{198.41.0.4}.

\begin{answer}{30em}
TODO
\end{answer}

\end{problem}

\clearpage
\begin{problem}

Suppose that you walked into Boelter Hall and get connected to \url{CSD} WiFi network, which automatically gave you IP address of the local caching resolver.
However, initially, it doesn't allow you to do anything unless you type your username and password in a popup window (or if you try to go to any website in your browser).

\begin{enumerate}

\item Explain a mechanism of how does the ``\url{CSD}'' network achieve this / which features of DNS/HTTP make it possible.

\end{enumerate}

After you successfully logged in, you can start using the Internet.  Suppose the caching resolver has just rebooted and its cache is completely empty;  RTT between your computer and the caching resolver is $10 ms$ and RTT between the caching resolver and any authoritative name server is $100 ms$; all responses have TTL 12 hours.

\begin{enumerate}\addtocounter{enumi}{1}
\item If you try to go to \url{ucla.edu}, what would be minimum amount of time you will need to wait before your web browser will be able to initiate connect to the UCLA's web server?

\item What would be the time, if a minute later you will decide to go to \url{ccle.ucla.edu}?

\item What would be the time, if another minute later you will decide to go to \url{piazza.com}?

\item What would be the time, if another minute later you will decide to go to \url{gradescope.com}?

\end{enumerate}

\begin{answer}{30em}
\end{answer}


\end{problem}


\clearpage
\begin{problem}

  \begin{enumerate}
  \item An online chatting application is going viral.
    To optimize user experience, its developers decided to use CDN service to deliver superb chat application performance for clients around the world.
    What mechanisms CDN services use to help developers to do so?
    In your answer include specific mechanisms and basic idea how these mechanisms work.

  \item What are the factors that go into designing a CDN server selection strategy? Name at least four.
  \end{enumerate}

\begin{answer}{24em}
\end{answer}

\end{problem}


\end{document}
