\documentclass{report}
\usepackage{homework}
\solstrue

\renewcommand{\hmwkTitle}{Homework 1}
\renewcommand{\hmwkAuthorName}{Rodrigo Valle -- 104 494 120}

\begin{document}
\mktitle

\begin{problem}


\newcommand{\ENDL}{$<$CR$>$$<$LF$>$}

Consider the following string of ASCII characters that were captured by
$Wireshark$ when the browser sent an HTTP GET message (i.e., this is the
actual content of an HTTP GET message). The characters \texttt{\ENDL{}} are
carriage-return and line-feed characters. Answer the following questions,
indicating where in the HTTP GET message below you find the answer.\\

\begin{raggedright}
\texttt{GET /classes/spring17/cs118/project-1.html HTTP/1.1\ENDL\\
Host: web.cs.ucla.edu\ENDL\\
Connection: keep-alive\ENDL\\
Upgrade-Insecure-Requests: 1\ENDL\\
User-Agent: Mozilla/5.0 (Macintosh; Intel Mac OS X 10\_12\_3) AppleWebKit/537.36 (KHTML, like Gecko) Chrome/56.0.2924.87 Safari/537.36\ENDL\\
Accept: text/html,application/xhtml+xml,application/xml;q=0.9,image/webp,*/*;q=0.8\ENDL\\
Referer: http://web.cs.ucla.edu/classes/spring17/cs118/homeworks.html\ENDL\\
Accept-Encoding: gzip, deflate, sdch\ENDL\\
Accept-Language: en-US,en;q=0.8,lv;q=0.6,ru;q=0.4\ENDL\\
If-None-Match: "5a17-54c4847c4f640-gzip"\ENDL\\
If-Modified-Since: Mon, 03 Apr 2017 19:36:49 GMT\ENDL\\
}
\end{raggedright}

\begin{enumerate}
\item What is the \textbf{full} URL of the document requested by the browser?
\item What version of HTTP is the browser running?
\item What type of browser initiates this message? Why is the browser type
      needed in an HTTP  request message?
\item What is the IP Address of the host on which the browser is running?
\end{enumerate}

\begin{answer}{30em}
  \begin{enumerate}
  \item http://web.cs.ucla.edu/classes/spring17/cs118/project-1.html
  \item From the end of the first line: \texttt{HTTP/1.1}
  \item From the \texttt{User-Agent} header, with a little research:
        \texttt{Chrome 56}.\\
        This message is useful to the application developer if the developer
        wishes to support multiple browsers that may not adhere to the
        standard. The server can parse this User-Agent to identify the
        type of browser and serve a modified version of the app which was
        tested on that browser.

        This link was helpful in answering this question:\\
        \url{https://developer.mozilla.org/en-US/docs/Web/HTTP/Browser_detection_using_the_user_agent}
  \item These HTTP headers do not provide us with that information; in fact,
        it's a rather odd question to ask at this level because the
        source/destination IP address is what the IP layer uses in the IP
        header to deliver the individual packets that compose this HTTP
        request.
  \end{enumerate}
\end{answer}
\end{problem}

\begin{problem}

Alice and Bob are working remotely on a course project and are using
\texttt{git} as the version control software.

\begin{enumerate}
\item Is it true that one must have GitHub/GitLab account to use git?
\item What is(are) the command(s) to initialize a local git repository?
\item Do Alice and Bob both must initialize local git repository? If no, what
      are the alternative?
\item Let's consider that Alice modified the file \texttt{server.cpp}:
  \begin{enumerate}
  \item What git commands Alice needs to save modifications in the local git
        repository
  \item What git commands Alice needs to upload saved modifications to GitHub
  \item What git commands Bob needs to get Alice's changes and apply them to
        the local repository
  \end{enumerate}
\item Let's consider that both Alice and Bob modified the file
      \texttt{server.cpp} and Alice was first to successfully upload saved
      modifications (commit) to GitHub
  \begin{enumerate}
  \item Can Bob upload his changes without any additional actions? If no, why?
  \item If actions needed, list git commands that Bob will need to use to
        share his modifications with Alice.
  \end{enumerate}
\end{enumerate}

\begin{answer}{37em}
  \begin{enumerate}
  \item It is absolutely false that you must have a GitHub/GitLab account to
        use git. They are different tools; git is used for version control and
        GitHub/GitLab are hosting solutions for git repositories.
  \item The command to initialize a local git repository in your current
        working directory is \texttt{git init}
  \item Alice and Bob probably shouldn't both initialize a local git
        repository unless they want to create a lot more work for themselves.
        It's much easier for one of them (e.g. Alice) to set up the
        respository on GitHub (e.g. at the URL
        \texttt{http://github.com/Alice/repository.git}) and for the other
        (e.g. Bob) to run the command
        \texttt{git clone http://github.com/Alice/respository.git}
  \item
    \begin{enumerate}
    \item To save modifications in the local repository, Alice should run the
          commands:\\
          \texttt{
            git add server.txt \# adds server.txt to commit staging\\
            git commit \# creates a new commit which stores the changes
          }
    \item To upload modifications to GitHub, Alice needs to create a
          repository on GitHub first. Let's say the URL is
          \texttt{http://github.com/Alice/repository.git}\\
          \texttt{
            git remote add origin http://github.com/Alice/repository.git\\
            git push origin master
          }
    \item To get Alice's changes, Bob must run the \texttt{git pull} command
          which will require him to merge his local commits with the commits
          on GitHub's servers (if he's made any to the current branch).
    \end{enumerate}
  \item
    \begin{enumerate}
      \item Bob can't upload his changes to GitHub without additional action
            because his changes are in conflict with Alice's changes on
            GitHub's server -- git has two versions of what
            \texttt{server.cpp} should look like and we need to specify which
            changes will make it to the ``final'' version of
            \texttt{server.cpp} (this process is known as merging).
      \item To share his changes, Bob must run \texttt{git pull} which will
            require him to merge his changes with Alice's (creating a new
            merge commit that references two parents -- Alice's last commit
            and Bob's last commit). After the merge is complete, Bob will be
            able to push his merged changes onto the GitHub server.
    \end{enumerate}
  \end{enumerate}
\end{answer}

\end{problem}


\begin{problem}

You will learn some basic usages of \texttt{Vagrant} and \texttt{Docker} in your projects.

\begin{enumerate}
\item What is Vagrant mainly used for?
\item What is Docker mainly used for?
\item List at least one difference between Vagrant's ``box'' and Docker's
      ``container''?
\item List at least five commands you can use with Vagrant.
\item List at least five commands you can use with Docker.
\end{enumerate}

\begin{answer}{35em}
  \begin{enumerate}
    \item Vagrant is used to create development environments running on
          virtual machines that are identical to the environment used in
          production for ease of testing, development, and potentially
          deployment. The VM is actually simulated by a provider like VMWare,
          Virtualbox, etc.
    \item Docker is different from Vagrant; it's used to wrap an application
          in a filesystem/execution environment that has everything the app
          needs to run correctly. The idea is that this ``container'' is
          standardized and can be ``shipped'' and executed on different
          machines, where the application will run in the same environment
          regardless of physical machine.
    \item The difference between Vagrant's box and Docker's container is how
          they abstract the application environment: Vagrant relies on
          virtualization, whereas Docker relies on containerization.
          Virtualization requires the emulation of an operating system on
          simulated hardware and is in a computational sense more expensive to
          do than containerization. With containerization, the application
          runs directly on the host OS and hardware, but in an execution
          environment that is isolated from the rest of the OS through
          specially designed software tools.
    \item
      \begin{enumerate}
        \item \texttt{vagrant init \# creates Vagrantfile}
        \item \texttt{vagrant up \# starts and sets up VM environment}
        \item \texttt{vagrant ssh \# access the VM}
        \item \texttt{vagrant halt \# stop the VM}
        \item \texttt{vagrant reload \# restart the VM with new configuration}
      \end{enumerate}
    \item
      \begin{enumerate}
        \item \texttt{docker ps \# list running containers}
        \item \texttt{docker build \# build Docker image from Dockerfile}
        \item \texttt{docker run \# run a container}
        \item \texttt{docker attach \# enter a running container}
        \item \texttt{docker kill \# stop a running container}
      \end{enumerate}
  \end{enumerate}
\end{answer}

\end{problem}

\end{document}
